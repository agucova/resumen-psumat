\section{Álgebra y Funciones}
\def\svgwidth{\columnwidth}

\subsection{Factorización}
$ (a \pm b \pm c)^2 = a^2 + b^2 + c^2 \pm 2ab \pm 2ac \pm 2bc $\\
$(a \pm b)^3 = a^3 \pm 3a^2b + 3ab^2 \pm b^3$\\
$ a^3 \pm b^3 = (a \pm b)(a^2 \mp ab + b^2)$\\
$ (x + p)(x + q) = x^2 + x(p + q) + pq $\\
$ a (a + b + 1) = a^2 + ab + a $\\

\subsection{Inyectividad y Epiyectividad}
Una función $f(x)$ es \textit{inyectiva} cuando,
$\forall a,b \in X, \;\; f(a)=f(b) \Rightarrow a=b$
, es decir cuando nunca mapea elementos distintos de su \textit{Dominio} a un mismo elemento del \textit{Codominio}.

Análogamente, se dice que una función f(x) es \textit{sobreyectiva}, o epiyectiva cuando, $\forall y \in Y, \, \exists x \in X, \;\; f(x)=y$, es decir, que para cada elemento $y$ en el codominio $Y$, hay al menos un elemento $x$ en el dominio $X$ de forma que $f(x) = y$.\\
\includegraphics[width=\columnwidth]{inyectividad}\\
\includegraphics[width=\columnwidth]{sobreyectividad}
\subsection{Función Afín}
Una función afin tiene una \textbf{forma principal} de $y = mx + n$, donde se denomina lineal si $n=0$, y una \textbf{forma general} de $ax + by = 0$.\\
Para la forma general, $m = -\frac{a}{b}$ y $n = -\frac{c}{b}.$\\
Punto pendiente: $y - y_1 = m(x - x_1)$\\
Dos Puntos: $\frac{y-y_1}{x-x_1} = \frac{y_2-y_1}{x_2-x_1}$\\
Distancia punto-recta: $\frac{|ax_0+by_0+c|}{\sqrt{a^2+b^2}}.$\\
\textit{Nota:} $m = \tan \alpha$

\subsection{Función Cuadrática}
Formatos:\\
$f(x) = a x^2 + b x + c \,\!$ llamada \textbf{forma estándar}\\
$f(x) = a(x - r_1)(x - r_2)\,\!$, llamada \textbf{forma factorizada}, con $r_1$ y $r_2$ raíces.\\
$f(x) = a(x - h)^2 + k \,\!$, llamada \textbf{forma de vértice}, con un vértice $(h, k)$.\\

Vértice: $(\frac{-b}{2a}),-\frac{b^2-4ac}{4a})$\\
$x_1 + x_2 = \frac{-b}{a}$\\
$x_1 \cdot x_2 = \frac{c}{a}$\\

\subsection{Función Exponencial}
Una función exponencial usualmente está descrita en la forma $(x) = ab^x$ con b un número real positivo y $x$ exponente. De forma simplificada $f(x) = b^x$, donde si $b > 1$ la función es creciente (hacia la derecha) y cuando $0 < b < 1$ la función es decreciente.\\

\subsection{Función Potencia}
Clase de la que pertenece la función cuadrática, está definida en forma general por $f(x) = ax^n$ con $n \in \mathbb{N} - \{1\}$ y $a \in \mathbb{R}$.

Cuando $n$ es \textbf{par}, entonces la función toma la forma de una parábola, y cuando es \textbf{impar}, toma una forma similar a la cúbica.

\subsection{Función Logarítmica}
Siguen la forma general $f(x) = log_b{x}$, y son inversas de las funciones exponenciales, siendo simétricas con respecto a $y = x$.\\
\begin{minipage}[c]{\columnwidth}
    \includegraphics[width=0.49\columnwidth]{funcionlogaritmica1}
    \includegraphics[width=0.49\columnwidth]{funcionlogaritmica2}
\end{minipage}